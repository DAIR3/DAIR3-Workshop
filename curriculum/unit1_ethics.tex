%%%%%%%%%%%%%%%%%%%%%%%%%%%%%%%%%%%%%%%%%%%%%%%%%%%%%%%%%%%%%%%%%
\chapter{Unit 1: Responsible Conduct of Research}
\label{chap:unit1}
%%%%%%%%%%%%%%%%%%%%%%%%%%%%%%%%%%%%%%%%%%%%%%%%%%%%%%%%%%%%%%%%%

\textbf{Total Time: 3 hours}

%================================================================
\section{RCR in the Context of Biomedical Data Science}
\label{sec:1.1}
%================================================================

This lesson examines the sociotechnical and ethical aspects of biomedical data science. We will consider ethical issues in the responsible conduct of research that are novel to or pose new challenges in the context of biomedical data science such as reproducibility and privacy. Students will also consider biomedical data science as a sociotechnical system and define roles for themselves and other key constituents.

\subsection{Learning Objectives}

\begin{enumerate}
    \item Explain novel ethical issues in responsible conduct of research for data science such as reproducibility and privacy.
    \item Describe the landscape of biomedical data science as a sociotechnical system and articulate roles.
\end{enumerate}

\subsection{Assessment Instrument}

\begin{itemize}
    \item Compare responses to the data challenge with your peers. What issues arise?
    \item Why was date and time no longer recorded after 1987?
    \item Who decides what data to collect, how to store it and how to access it? What biases could there be in the data (e.g., data collection in rural areas, existence of infrastructure)? What is the difference between bias and trend in these data? Provide examples.
\end{itemize}

%================================================================
\section{What are Ethics? Ethical Issues in Biomedical Data Science}
\label{sec:1.2}
%================================================================

This lesson equips students to address ethical challenges in biomedical data science. Learners will identify strategies for ethical secondary data use, analyze engagement approaches, and develop frameworks for ethical project review, emphasizing anticipatory governance and responsible data science practices. Case studies will be used that draw on the group project selected for the 2026 cohort.

\subsection{Learning Objectives}

\begin{enumerate}
    \item Differentiate between traditional bioethical, sociotechnical, and other ethical approaches to data science research and applications.
    \item Evaluate key ethical challenges in biomedical data science.
    \item Identify and formulate approaches to address ethical issues in secondary use, including anticipatory governance principles.
    \item Develop a framework for ethical review of biomedical data science projects.
\end{enumerate}

\subsection{Assessment Instrument}

The assessment involves designing a governance framework for a data science initiative, selecting one of three projects derived from the course-wide data activity. The framework must address stakeholder engagement, decision-making, monitoring, benefit-sharing, unexpected impacts, and consent, while evaluating ethical considerations, future challenges, and feasibility. The goal is to create an ethical, well-structured, and adaptable governance plan.


\begin{landscape}
    %================================================================
    \section{Rubric}
    %================================================================
    {\footnotesize

    \begin{longtable}{>{\RaggedRight\arraybackslash}p{2.5cm} >{\RaggedRight\arraybackslash}p{3cm} >{\RaggedRight\arraybackslash}p{2.8cm} >{\RaggedRight\arraybackslash}p{2.8cm} >{\RaggedRight\arraybackslash}p{3cm}}
    \toprule
    \textbf{Component} & \textbf{Excellent (90--100\%)} & \textbf{Good (80--89\%)} & \textbf{Satisfactory (70--79\%)} & \textbf{Needs Improvement ($<$70\%)} \\
    \midrule
    \endfirsthead

    \toprule
    \textbf{Component} & \textbf{Excellent (90--100\%)} & \textbf{Good (80--89\%)} & \textbf{Satisfactory (70--79\%)} & \textbf{Needs Improvement ($<$70\%)} \\
    \midrule
    \endhead

    \midrule
    \multicolumn{5}{r}{\textit{Continued on next page}} \\
    \endfoot

    \bottomrule
    \endlastfoot

    \textbf{1. Metadata -- Data About Data} \newline \textbf{(20\%)} 
    & Identifies 5 critical metadata categories with clear, relevant examples; provides a thorough analysis of metadata from Hauer's article, including how, on whom, and under what circumstances data were collected; differentiates high-quality vs poor metadata for reproducibility. 
    & Lists most metadata categories with relevant examples; analysis from Hauer article is solid but lacks some detail; distinction between good/bad metadata is mostly clear. 
    & Lists some metadata categories but with limited examples or partial relevance; Hauer's article analysis is basic; distinction is made at the surface level. 
    & Fewer than 5 categories or inaccurate examples; superficial or missing analysis of Hauer article; shows little grasp of metadata quality for reproducibility. \\
    \midrule

    \textbf{2. Data Representation} \newline \textbf{(20\%)} 
    & Clearly describes alternative data representations and justifies which schema best supports analysis tasks; insightfully compares the flat file to alternative, with thoughtful reasoning and examples; demonstrates a strong understanding of how representation affects research. 
    & Describes an alternative schema and makes a reasonable comparison to the flat file, with some justification; demonstrates good understanding with minor gaps. 
    & Identifies a basic alternative, but justification/comparison to the flat file is weak or incomplete; limited examples. 
    & Incomplete, irrelevant, or unclear alternatives; little to no justification or analysis of schema versus flat file; shows limited understanding. \\
    \midrule

    \textbf{3. Data Sharing 101: Open Science, NIH Policy, FAIR} \newline \textbf{(30\%)} 
    & Thoroughly defines rationale for NIH Data Management \& Sharing; lists 5 key components with well-explained relevance; accurately lists and explains all 4 FAIR principles with practical examples for each. 
    & Defines rationale and lists the key NIH components with basic relevance; lists and explains all FAIR principles, with most examples being suitable. 
    & Provides basic rationale and fewer than 5 NIH components; lists most FAIR principles with limited or superficial examples. 
    & Incomplete rationale and components, missing several critical points; fails to list all FAIR principles or provides incorrect explanations/examples. \\
    \midrule

    \textbf{4. Data Sharing (The Reality): Privacy, Confidentiality, Types of Biomedical Research} \newline \textbf{(30\%)} 
    & Creates a complete Data Management \& Sharing Plan tailored to the Texas birthweight challenge, following NIH requirements; thoroughly applies FAIR principles to the dataset; clearly articulates privacy concerns and differences among bench, clinical, and animal model research, with thoughtful analysis of sharing implications. 
    & Plan meets most NIH requirements; applies FAIR principles with minor gaps; reasonably discusses privacy and research type differences. 
    & Plan basic or incomplete; applies FAIR principles at the surface level; discusses privacy/research types with limited detail. 
    & Missing/incomplete plan; fails to address FAIR principles, research types, or privacy concerns meaningfully. \\

    \end{longtable}
    }

\end{landscape}