%%%%%%%%%%%%%%%%%%%%%%%%%%%%%%%%%%%%%%%%%%%%%%%%%%%%%%%%%%%%%%%%%
\chapter{Unit 5: Reproducible Workflows}
\label{chap:unit5}
%%%%%%%%%%%%%%%%%%%%%%%%%%%%%%%%%%%%%%%%%%%%%%%%%%%%%%%%%%%%%%%%%

\textbf{Total Time: 5.5 hours}

%================================================================
\section{Goals of Reproducible Analyses}
\label{sec:5.1}
%================================================================

Learn the key goals and challenges of creating reproducible, transparent, and user-friendly analyses that are easy to share and reuse.

\subsection{Learning Objectives}

\begin{enumerate}
    \item Awareness of key challenges and goals when creating reproducible workflows, including making analyses reproducible, user friendly, transparent, reusable, version controlled, and archived.
\end{enumerate}

%================================================================
\section{Reproducibility via Code Notebooks}
\label{sec:5.2}
%================================================================

Gain awareness of Markdown, Jupyter, and Quarto, and learn how these tools integrate to create clear, reproducible workflows for data analysis and reporting.

\subsection{Learning Objectives}

\begin{enumerate}
    \item Awareness of Markdown, Jupyter, Quarto, and how these tools can be integrated into reproducible workflows.
\end{enumerate}

%================================================================
\section{Best Practices for Reproducible Programming}
\label{sec:5.3}
%================================================================

Learn essential best practices for reproducible programming, including writing clear scripts and functions, avoiding magic numbers, using caching and seeding for randomness, and refactoring code to enhance clarity, reliability, and repeatability.

\subsection{Learning Objectives}

\begin{enumerate}
    \item Awareness of best practices for reproducible programming including writing scripts, functions, avoiding magic numbers, caching and seeding randomness, and how to refactor code to align with these practices.
\end{enumerate}

%================================================================
\section{Version Control}
\label{sec:5.4}
%================================================================

Gain a basic understanding of Git, its advantages, and learn to perform essential tasks such as cloning repositories, committing changes, and syncing with remote repositories using push and pull commands.

\subsection{Learning Objectives}

\begin{enumerate}
    \item Familiarity with Git and its benefits, and the ability to begin using it for simple tasks, including cloning, committing changes, pushing and pulling.
\end{enumerate}

%================================================================
\section{Containers}
\label{sec:5.5}
%================================================================

Gain hands-on experience with key dependency management tools (Python virtual environments, renv, and containerization), understanding their pros and cons, and develop the skills to create and run basic Docker images.

\subsection{Learning Objectives}

\begin{enumerate}
    \item Familiarity with various tools for dependency management, including Python virtual environments, renv, and containerization, and their respective strengths and weaknesses. Ability to create and run simple Docker images.
\end{enumerate}

%================================================================
\section{Assembling a Full Analysis Pipeline}
\label{sec:5.6}
%================================================================

Learn key factors in organizing an analysis pipeline and develop the skills to assemble a complete, reusable pipeline template.

\subsection{Learning Objectives}

\begin{enumerate}
    \item Considerations when organizing an analysis pipeline, and the ability to assemble a full template pipeline.
\end{enumerate}

\subsection{Assessment Instrument}

\begin{enumerate}
    \item Describe your progress on the template workflow. What aspects did you find most confusing or challenging? Which tools (e.g., Git, Make, Docker) were hardest to implement, and why?
    \item What is one thing you plan to change or do differently in your own projects after today's session? Give a specific example of an analysis or workflow improvement you intend to make.
\end{enumerate}

