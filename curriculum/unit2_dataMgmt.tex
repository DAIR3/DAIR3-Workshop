%%%%%%%%%%%%%%%%%%%%%%%%%%%%%%%%%%%%%%%%%%%%%%%%%%%%%%%%%%%%%%%%%
\chapter{Unit 2: Data Management}
\label{chap:unit2}
%%%%%%%%%%%%%%%%%%%%%%%%%%%%%%%%%%%%%%%%%%%%%%%%%%%%%%%%%%%%%%%%%

\textbf{Total Time: 7 hours}


%================================================================
\section{Data Collection}
\label{sec:DataCollection}
%================================================================

Pending... 

%================================================================
\section{Metadata - Data About Data}
\label{sec:2.2}
%================================================================

\textbf{Time: 0.75 hours}

This lesson explores the essentials of metadata in biomedical research datasets, covering data collection methods, population, and context. Students will learn to identify high-quality metadata that supports reproducibility and distinguish it from inadequate metadata, ensuring robust and reliable research outcomes.

\subsection{Learning Objectives}

\begin{enumerate}
    \item Understand standard components of metadata on biomedical science research datasets: how the data were collected, on what population, under what circumstances, etc.
    \item Learn to distinguish between good and bad metadata for reproducibility.
\end{enumerate}

\subsection{Assessment Instrument}

\begin{itemize}
    \item List 5 critical metadata categories that researchers need to know when reproducing findings.
    \item From the Mathew E. Hauer article called, ``Data Descriptors: Population projections for U.S. Counties by age, sex, and race,'' what metadata are provided about how the population data were collected?
\end{itemize}


%================================================================
\section{Data Representation}
\label{sec:2.3}
%================================================================

\textbf{Time: 0.75 hours}

This lesson examines how data can be represented in multiple ways, highlighting that each representation impacts task efficiency. Students will learn to select optimal data representations tailored to specific research tasks, balancing ease and complexity.

\subsection{Learning Objectives}

\begin{enumerate}
    \item Understand that the same data can be represented in many ways.
    \item Appreciate that each representation choice makes some tasks easier, but others more difficult.
    \item Learn how to choose a good representation for the task at hand.
\end{enumerate}

\subsection{Assessment Instrument}

The NCHS data are provided as a flat file with more than 100 variables. What is an alternative representation of these same data? Is the original flat file or your alternative schema more conducive to analyses, and why?

%================================================================
\section{Data Sharing}
\label{sec:2.4}
%================================================================

\textbf{Time: 2.5 hours}

\subsubsection{Data Sharing 101}

This lesson introduces the principles of Open Science, focusing on the NIH Data Management \& Sharing Policy's rationale and key components. Students will explore the FAIR Guiding Principles (Findable, Accessible, Interoperable, Reusable), learning their definitions and practical examples to promote transparent and reproducible research.

\textbf{Learning Objectives:}
\begin{itemize}
    \item Appreciate the foundations of Open Science
    \item NIH Data Management \& Sharing Policy: Rationale and Key components
    \item FAIR Guiding Principles: Definition and examples
\end{itemize}

\textbf{Assessment:} Define rationale behind NIH Data Management \& Sharing requirement; list 5 key components you should include in your 2-page Data Management Plan; list the 4 FAIR principles.

\subsubsection{Data Sharing - The Reality}

This lesson examines privacy and confidentiality concerns in Open Science and data sharing. Students will learn to differentiate between biomedical research types: bench science, human clinical trials, and animal models, and understand the unique data sharing implications for each, including ethical considerations and strategies to protect sensitive data while promoting transparency.

\textbf{Learning Objectives:}
\begin{enumerate}
    \item Learn about privacy/confidentiality concerns related to Open Science and data sharing.
    \item Articulate difference in types of biomedical research (bench science, human clinical trials, animal models) and what implications data sharing has for each.
\end{enumerate}

\textbf{Assessment:}
\begin{itemize}
    \item Create a 2-page Data Management and Sharing Plan following the NIH requirements for your analyses of the birthweight data challenge.
    \item Apply the FAIR principles (Findable, Accessible, Interoperable, Reusable) to your datasets.
\end{itemize}

% Required packages (add to your preamble if not already there):
% \usepackage{longtable}
% \usepackage{booktabs}
% \usepackage{ragged2e}
% \usepackage{array}
% \usepackage{pdflscape}

\begin{landscape}

    \section{Evaluation Rubric}

    {\footnotesize

    \begin{longtable}{%
        >{\RaggedRight\arraybackslash}p{2.5cm}   % Component
        >{\RaggedRight\arraybackslash}p{4.5cm}   % Excellent
        >{\RaggedRight\arraybackslash}p{4.2cm}   % Good
        >{\RaggedRight\arraybackslash}p{4.2cm}   % Satisfactory
        >{\RaggedRight\arraybackslash}p{4.5cm}}  % Needs Improvement

    \toprule
    \textbf{Component} &
    \textbf{Excellent (90--100\%)} &
    \textbf{Good (80--89\%)} &
    \textbf{Satisfactory (70--79\%)} &
    \textbf{Needs Improvement ($<$70\%)} \\
    \midrule
    \endfirsthead

    \toprule
    \textbf{Component} &
    \textbf{Excellent (90--100\%)} &
    \textbf{Good (80--89\%)} &
    \textbf{Satisfactory (70--79\%)} &
    \textbf{Needs Improvement ($<$70\%)} \\
    \midrule
    \endhead

    \midrule
    \multicolumn{5}{r}{\textit{Continued on next page}} \\
    \endfoot

    \bottomrule
    \endlastfoot

    % Row 1
    \textbf{1.\ Metadata -- Data About Data}
    \newline\textbf{(20\%)}
    &
    Identifies 5 critical metadata categories with clear, relevant examples;
    provides a thorough analysis of metadata from Hauer's article, including
    how, on whom, and under what circumstances data were collected;
    differentiates high-quality vs.\ poor metadata for reproducibility.
    &
    Lists most metadata categories with relevant examples; analysis from the
    Hauer article is solid but lacks some detail; distinction between
    good/bad metadata is mostly clear.
    &
    Lists some metadata categories but with limited examples or partial
    relevance; Hauer's article analysis is basic; distinction is made at the
    surface level.
    &
    Fewer than 5 categories or inaccurate examples; superficial or missing
    analysis of Hauer article; shows little grasp of metadata quality for
    reproducibility.
    \\
    \midrule

    % Row 2
    \textbf{2.\ Data Representation}
    \newline\textbf{(20\%)}
    &
    Clearly describes alternative data representations and justifies which
    schema best supports analysis tasks; insightfully compares the flat file
    to the alternative, with thoughtful reasoning and examples; demonstrates
    a strong understanding of how representation affects research.
    &
    Describes an alternative schema and makes a reasonable comparison to the
    flat file, with some justification; demonstrates good understanding with
    minor gaps.
    &
    Identifies a basic alternative, but justification/comparison to the flat
    file is weak or incomplete; limited examples.
    &
    Incomplete, irrelevant, or unclear alternatives; little to no
    justification or analysis of schema versus flat file; shows limited
    understanding.
    \\
    \midrule

    % Row 3
    \textbf{3.\ Data Sharing 101: Open Science, NIH Policy, FAIR}
    \newline\textbf{(30\%)}
    &
    Thoroughly defines rationale for NIH Data Management \& Sharing; lists 5
    key components with well-explained relevance; accurately lists and
    explains all 4 FAIR principles with practical examples for each.
    &
    Defines rationale and lists the key NIH components with basic relevance;
    lists and explains all FAIR principles, with most examples being suitable.
    &
    Provides basic rationale and fewer than 5 NIH components; lists most
    FAIR principles with limited or superficial examples.
    &
    Incomplete rationale and components, missing several critical points;
    fails to list all FAIR principles or provides incorrect
    explanations/examples.
    \\
    \midrule

    % Row 4
    \textbf{4.\ Data Sharing (The Reality): Privacy, Confidentiality,
    Types of Biomedical Research}
    \newline\textbf{(30\%)}
    &
    Creates a complete Data Management \& Sharing Plan tailored to the Texas
    birthweight challenge, following NIH requirements; thoroughly applies
    FAIR principles to the dataset; clearly articulates privacy concerns and
    differences among bench, clinical, and animal model research, with
    thoughtful analysis of sharing implications.
    &
    Plan meets most NIH requirements; applies FAIR principles with minor
    gaps; reasonably discusses privacy and research type differences.
    &
    Plan basic or incomplete; applies FAIR principles at the surface level;
    discusses privacy/research types with limited detail.
    &
    Missing/incomplete plan; fails to address FAIR principles, research
    types, or privacy concerns meaningfully.
    \\

    \end{longtable}

    } % end \footnotesize

\end{landscape}