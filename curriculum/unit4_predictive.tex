%%%%%%%%%%%%%%%%%%%%%%%%%%%%%%%%%%%%%%%%%%%%%%%%%%%%%%%%%%%%%%%%%
\chapter{Unit 4: Designing Interpretable Predictive Models}
\label{chap:unit4}
%%%%%%%%%%%%%%%%%%%%%%%%%%%%%%%%%%%%%%%%%%%%%%%%%%%%%%%%%%%%%%%%%

\textbf{Total Time: 5 hours} \\
\textbf{Instructional Time: 3.5 hours} \\
\textbf{Unit Project Work Time: 1.5 hours}

This unit will introduce the foundations of supervised machine learning models, with a focus on interpretability and communicating decision-making.

%================================================================
\section{Pre-reading Materials}
\label{sec:4.0}
%================================================================

These pre-reading materials focus on foundational concepts in statistics and machine learning.

\subsection{Learning Objectives}

Learners will be able to:
\begin{enumerate}
    \item Understand ideas from introductory statistics, including measures of central tendency and variability, visualization techniques, and lines of best fit.
    \item Understand how to fit a basic machine learning model in Python using sklearn and real data.
    \item Understand TRIPOD guidelines pertinent to this session.
\end{enumerate}

\subsection{Assessment Instrument}

Learners will be asked to load a dataset into sklearn, fit a linear regression model, and report the model's mean squared error.

%================================================================
\section{Foundations of Supervised Learning}
\label{sec:4.1}
%================================================================

\textbf{Time: 1.5 hours (Instructional: 70 minutes)}

We will start by providing a broad overview of the landscape of machine learning, and practice the general workflow for building a model, starting from raw data.

\subsection{Learning Objectives}

\begin{enumerate}
    \item Understand the landscape of possible machine learning models (supervised vs. unsupervised, regression vs. classification).
    \item Build a linear regression model, understanding how its optimal parameters were chosen and how they can be interpreted.
    \item Practice the workflow for performing a train-test split, training a model on training data, evaluating a model on held-out test data, and the role of cross-validation.
    \item Understand the risks of overfitting and data leakage.
\end{enumerate}

\subsection{Assessment Instrument (20 minutes)}

Learners will build a basic linear regression model using the Vital Statistics dataset, which will serve as a baseline for future work.

%================================================================
\section{Feature Engineering}
\label{sec:4.2}
%================================================================

\textbf{Time: 1 hour (Instructional: 60 minutes)}

Next, we will cover how linear regression can be extended to a variety of other tasks, and how to create new features that capture trends in the data.

\subsection{Learning Objectives}

\begin{enumerate}
    \item Practice interpreting the coefficients of fit models.
    \item Use visualizations to spot patterns in the data that inform feature engineering decisions, while keeping in mind the risks of overfitting.
    \item Understand the differences between encoding strategies (one hot encoding vs. ordinal encoding).
\end{enumerate}

\subsection{Assessment Instrument (20 minutes)}

Learners will be given a practical task and will need to build a small feature engineering Pipeline in sklearn and be asked to document their decision-making process.

%================================================================
\section{Feature Selection and Model Explainability}
\label{sec:4.3}
%================================================================

\textbf{Time: 1.5 hours (Instructional: 60 minutes)}

Building upon Section~\ref{sec:4.2}, students will gain an understanding of the statistical approaches involved in selecting features.

\subsection{Learning Objectives}

\begin{enumerate}
    \item Understand mutual information as a feature selection criterion.
    \item Understand how to apply statistically appropriate techniques to select and justify features, e.g., Pearson, Spearman, and Cram\'er's V correlations, Variance Inflation Factor, multiple $R^2$.
    \item Understand the role of interaction terms.
\end{enumerate}

\subsection{Assessment Instrument (30 minutes)}

Learners will practice generating feature importance analyses and documenting their feature selection rationale.

%================================================================
\section{Model Evaluation, Comparison, and Reporting}
\label{sec:4.4}
%================================================================

\textbf{Time: 1 hour (Instructional: 30 minutes)}

Finally, students will practice communicating the performance and design decisions behind a model to external stakeholders.

\subsection{Learning Objectives}

\begin{enumerate}
    \item Understand how to report and compare different models on the same task (e.g., MSE/RMSE/MAE for regression models, accuracy vs. precision vs. recall vs. ROC-AUC vs. F1 for classification models).
    \item Practice communicating modeling choices and model behavior.
\end{enumerate}

\subsection{Assessment Instrument (30 minutes)}

Students will write a short report documenting and comparing the performances of two models, including relevant visualizations.

